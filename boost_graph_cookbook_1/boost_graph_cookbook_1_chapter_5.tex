%%%%%%%%%%%%%%%%%%%%%%%%%%%%%%%%%%%%%%%%%%%%%%%%%%%%%%%%%%%%%%%%%%%%%%%%%%%%%%%%
\chapter{Working on graphs with bundled vertices}
\label{sec:Working-on-graphs-with-bundled-vertices}
%%%%%%%%%%%%%%%%%%%%%%%%%%%%%%%%%%%%%%%%%%%%%%%%%%%%%%%%%%%%%%%%%%%%%%%%%%%%%%%%

When using graphs with bundled vertices, their state gives a way to find
a vertex and working with it.
This chapter shows some basic operations on graphs with bundled vertices.

\begin{itemize}
  \item Check if there exists a vertex with a certain \verb;my_bundled_vertex;: 
    chapter \ref{subsec:has_bundled_vertex_with_my_vertex}
  \item Find a vertex with a certain \verb;my_bundled_vertex;: 
    chapter \ref{subsec:find_bundled_vertex_with_my_vertex}
  \item Get a vertex its \verb;my_bundled_vertex; from its vertex descriptor: 
    chapter \ref{subsec:get_bundled_vertex_my_vertex}
  \item Set a vertex its \verb;my_bundled_vertex; using its vertex descriptor: 
    chapter \ref{subsec:set_bundled_vertex_my_vertex}
  \item Setting all vertices their \verb;my_bundled_vertex;-es: 
    chapter \ref{subsec:set_bundled_vertex_my_vertexes}
  \item Storing an directed/undirected graph with bundled vertices as a .dot file: 
    chapter \ref{subsec:save_bundled_vertices_graph_to_dot}
  \item Loading a directed graph with bundled vertices from a .dot file: 
    chapter \ref{subsec:load_directed_bundled_vertices_graph_from_dot}
  \item Loading an undirected directed graph with bundled vertices from a .dot file:
    chapter \ref{subsec:load_undirected_bundled_vertices_graph_from_dot}
\end{itemize}

%%%%%%%%%%%%%%%%%%%%%%%%%%%%%%%%%%%%%%%%%%%%%%%%%%%%%%%%%%%%%%%%%%%%%%%%%%%%%%%%
\section{Has a bundled vertex with a my\_bundled\_vertex}
\label{subsec:has_bundled_vertex_with_my_vertex}
%%%%%%%%%%%%%%%%%%%%%%%%%%%%%%%%%%%%%%%%%%%%%%%%%%%%%%%%%%%%%%%%%%%%%%%%%%%%%%%%

Before modifying our vertices, let's first determine if we can find a vertex
by its bundled type (\verb;my_bundled_vertex;) in a graph.
After obtain the vertex iterators, we can dereference each these to obtain
the vertex descriptors and then compare each vertex its \verb;my_bundled_vertex;
with the one desired.

\lstinputlisting[
  caption = Find if there is vertex with a certain my\_bundled\_vertex,
  label = lst:has_bundled_vertex_with_my_vertex
]{has_bundled_vertex_with_my_vertex.impl}
\index{Has bundled vertex with my_vertex}

This function can be demonstrated as in algorithm 
\ref{lst:has_bundled_vertex_with_my_vertex_demo}, 
where a certain my\_bundled\_vertex cannot be found in an empty graph.
After adding the desired my\_bundled\_vertex, it is found.

\lstinputlisting[
  caption = Demonstration of the has\_bundled\_vertex\_with\_my\_vertex function,
  label = lst:has_bundled_vertex_with_my_vertex_demo
]{has_bundled_vertex_with_my_vertex_demo.impl}

Note that this function only finds if there is at least one bundled vertex
with that my\_bundled\_vertex: it does not tell how many bundled vertices
with that my\_bundled\_vertex exist in the graph.

%%%%%%%%%%%%%%%%%%%%%%%%%%%%%%%%%%%%%%%%%%%%%%%%%%%%%%%%%%%%%%%%%%%%%%%%%%%%%%%%
\section{Find a bundled vertex with a certain my\_bundled\_vertex}
\label{subsec:find_bundled_vertex_with_my_vertex}
%%%%%%%%%%%%%%%%%%%%%%%%%%%%%%%%%%%%%%%%%%%%%%%%%%%%%%%%%%%%%%%%%%%%%%%%%%%%%%%%

Where STL functions work with iterators, here we obtain a vertex descriptor
(see chapter \ref{subsec:Vertex-descriptors}) 
to obtain a handle to the desired vertex.
Listing \ref{lst:find_first_bundled_vertex_with_my_vertex}
shows how to obtain a vertex descriptor to the first vertex found with
a specific \verb;my_bundled_vertex; value.

\lstinputlisting[
  caption = Find the first vertex with a certain my\_bundled\_vertex,
  label = lst:find_first_bundled_vertex_with_my_vertex
]{find_first_bundled_vertex_with_my_vertex.impl}
\index{Find first bundled vertex with my_vertex}

With the vertex descriptor obtained, one can read and modify the vertex
and the edges surrounding it.
Listing 
\ref{lst:find_first_bundled_vertex_with_my_vertex_demo}
shows some examples of how to do so.

\lstinputlisting[
  caption = Demonstration of the find\_first\_bundled\_vertex\_with\_my\_vertex function,
  label = lst:find_first_bundled_vertex_with_my_vertex_demo
]{find_first_bundled_vertex_with_my_vertex_demo.impl}

%%%%%%%%%%%%%%%%%%%%%%%%%%%%%%%%%%%%%%%%%%%%%%%%%%%%%%%%%%%%%%%%%%%%%%%%%%%%%%%%
\section{Get a bundled vertex its my\_bundled\_vertex}
\label{subsec:get_bundled_vertex_my_vertex}
%%%%%%%%%%%%%%%%%%%%%%%%%%%%%%%%%%%%%%%%%%%%%%%%%%%%%%%%%%%%%%%%%%%%%%%%%%%%%%%%

To obtain the \verb;my_bundled_vertex; from a vertex descriptor is simple:

\lstinputlisting[
  caption = Get a bundled vertex its my\_vertex from its vertex descriptor,
  label = lst:get_bundled_vertex_my_vertex
]{get_my_bundled_vertex.impl}
\index{Get bundled vertex my_bundled_vertex}

One can just use the graph as a property map and let it be looked-up.

To use \verb;get_bundled_vertex_my_vertex;, 
one first needs to obtain a vertex descriptor.
Listing \ref{lst:get_bundled_vertex_my_vertex_demo}
shows a simple example.

\lstinputlisting[
  caption = Demonstration if the get\_bundled\_vertex\_my\_vertex function,
  label = lst:get_bundled_vertex_my_vertex_demo
]{get_my_bundled_vertex_demo.impl}

