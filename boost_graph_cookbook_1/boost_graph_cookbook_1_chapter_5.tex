%%%%%%%%%%%%%%%%%%%%%%%%%%%%%%%%%%%%%%%%%%%%%%%%%%%%%%%%%%%%%%%%%%%%%%%%%%%%%%%%
\chapter{Working on graphs with bundled vertices}
\label{sec:Working-on-graphs-with-bundled-vertices}
%%%%%%%%%%%%%%%%%%%%%%%%%%%%%%%%%%%%%%%%%%%%%%%%%%%%%%%%%%%%%%%%%%%%%%%%%%%%%%%%

When using graphs with bundled vertices, their state gives a way to find
a vertex and working with it.
This chapter shows some basic operations on graphs with bundled vertices.

\begin{itemize}
  \item Check if there exists a vertex with a certain \verb;my_bundled_vertex;: 
    chapter \ref{subsec:has_bundled_vertex_with_my_vertex}
  \item Find a vertex with a certain \verb;my_bundled_vertex;: 
    chapter \ref{subsec:find_bundled_vertex_with_my_vertex}
  \item Get a vertex its \verb;my_bundled_vertex; from its vertex descriptor: 
    chapter \ref{subsec:get_bundled_vertex_my_vertex}
  \item Set a vertex its \verb;my_bundled_vertex; using its vertex descriptor: 
    chapter \ref{subsec:set_bundled_vertex_my_vertex}
  \item Setting all vertices their \verb;my_bundled_vertex;-es: 
    chapter \ref{subsec:set_bundled_vertex_my_vertexes}
  \item Storing an directed/undirected graph with bundled vertices as a .dot file: 
    chapter \ref{subsec:save_bundled_vertices_graph_to_dot}
  \item Loading a directed graph with bundled vertices from a .dot file: 
    chapter \ref{subsec:load_directed_bundled_vertices_graph_from_dot}
  \item Loading an undirected directed graph with bundled vertices from a .dot file:
    chapter \ref{subsec:load_undirected_bundled_vertices_graph_from_dot}
\end{itemize}

%%%%%%%%%%%%%%%%%%%%%%%%%%%%%%%%%%%%%%%%%%%%%%%%%%%%%%%%%%%%%%%%%%%%%%%%%%%%%%%%
\section{Has a bundled vertex with a my\_bundled\_vertex}
\label{subsec:has_bundled_vertex_with_my_vertex}
%%%%%%%%%%%%%%%%%%%%%%%%%%%%%%%%%%%%%%%%%%%%%%%%%%%%%%%%%%%%%%%%%%%%%%%%%%%%%%%%

Before modifying our vertices, let's first determine if we can find a vertex
by its bundled type (\verb;my_bundled_vertex;) in a graph.
After obtain the vertex iterators, we can dereference each these to obtain
the vertex descriptors and then compare each vertex its \verb;my_bundled_vertex;
with the one desired.

\lstinputlisting[
  caption = Find if there is vertex with a certain my\_bundled\_vertex,
  label = lst:has_bundled_vertex_with_my_vertex
]{has_bundled_vertex_with_my_vertex.impl}
\index{Has bundled vertex with my_vertex}

This function can be demonstrated as in algorithm 
\ref{lst:has_bundled_vertex_with_my_vertex_demo}, 
where a certain my\_bundled\_vertex cannot be found in an empty graph.
After adding the desired my\_bundled\_vertex, it is found.

\lstinputlisting[
  caption = Demonstration of the has\_bundled\_vertex\_with\_my\_vertex function,
  label = lst:has_bundled_vertex_with_my_vertex_demo
]{has_bundled_vertex_with_my_vertex_demo.impl}

Note that this function only finds if there is at least one bundled vertex
with that my\_bundled\_vertex: it does not tell how many bundled vertices
with that my\_bundled\_vertex exist in the graph.

%%%%%%%%%%%%%%%%%%%%%%%%%%%%%%%%%%%%%%%%%%%%%%%%%%%%%%%%%%%%%%%%%%%%%%%%%%%%%%%%
\section{Find a bundled vertex with a certain my\_bundled\_vertex}
\label{subsec:find_bundled_vertex_with_my_vertex}
%%%%%%%%%%%%%%%%%%%%%%%%%%%%%%%%%%%%%%%%%%%%%%%%%%%%%%%%%%%%%%%%%%%%%%%%%%%%%%%%

Where STL functions work with iterators, here we obtain a vertex descriptor
(see chapter \ref{subsec:Vertex-descriptors}) 
to obtain a handle to the desired vertex.
Listing \ref{lst:find_first_bundled_vertex_with_my_vertex}
shows how to obtain a vertex descriptor to the first vertex found with
a specific \verb;my_bundled_vertex; value.

\lstinputlisting[
  caption = Find the first vertex with a certain my\_bundled\_vertex,
  label = lst:find_first_bundled_vertex_with_my_vertex
]{find_first_bundled_vertex_with_my_vertex.impl}
\index{Find first bundled vertex with my_vertex}

With the vertex descriptor obtained, one can read and modify the vertex
and the edges surrounding it.
Listing 
\ref{lst:find_first_bundled_vertex_with_my_vertex_demo}
shows some examples of how to do so.

\lstinputlisting[
  caption = Demonstration of the find\_first\_bundled\_vertex\_with\_my\_vertex function,
  label = lst:find_first_bundled_vertex_with_my_vertex_demo
]{find_first_bundled_vertex_with_my_vertex_demo.impl}

%%%%%%%%%%%%%%%%%%%%%%%%%%%%%%%%%%%%%%%%%%%%%%%%%%%%%%%%%%%%%%%%%%%%%%%%%%%%%%%%
\section{Get a bundled vertex its my\_bundled\_vertex}
\label{subsec:get_bundled_vertex_my_vertex}
%%%%%%%%%%%%%%%%%%%%%%%%%%%%%%%%%%%%%%%%%%%%%%%%%%%%%%%%%%%%%%%%%%%%%%%%%%%%%%%%

To obtain the \verb;my_bundled_vertex; from a vertex descriptor is simple:

\lstinputlisting[
  caption = Get a bundled vertex its my\_vertex from its vertex descriptor,
  label = lst:get_bundled_vertex_my_vertex
]{get_my_bundled_vertex.impl}
\index{Get bundled vertex my_bundled_vertex}

One can just use the graph as a property map and let it be looked-up.

To use \verb;get_bundled_vertex_my_vertex;, 
one first needs to obtain a vertex descriptor.
Listing \ref{lst:get_bundled_vertex_my_vertex_demo}
shows a simple example.

\lstinputlisting[
  caption = Demonstration if the get\_bundled\_vertex\_my\_vertex function,
  label = lst:get_bundled_vertex_my_vertex_demo
]{get_my_bundled_vertex_demo.impl}

%%%%%%%%%%%%%%%%%%%%%%%%%%%%%%%%%%%%%%%%%%%%%%%%%%%%%%%%%%%%%%%%%%%%%%%%%%%%%%%%
\section{Set a bundled vertex its my\_vertex}
\label{subsec:set_bundled_vertex_my_vertex}
%%%%%%%%%%%%%%%%%%%%%%%%%%%%%%%%%%%%%%%%%%%%%%%%%%%%%%%%%%%%%%%%%%%%%%%%%%%%%%%%

If you know how to get the \verb;my_bundled_vertex; from a vertex descriptor,
setting it is just as easy, 
as shown in algorithm \ref{lst:set_bundled_vertex_my_vertex}

\lstinputlisting[
  caption = Set a bundled vertex its my\_vertex from its vertex descriptor,
  label = lst:set_bundled_vertex_my_vertex
]{set_my_bundled_vertex.impl}
\index{Set vertex my\_vertex}

To use \verb;set_bundled_vertex_my_vertex;, 
one first needs to obtain a vertex descriptor.
Listing \ref{lst:set_bundled_vertex_my_vertex_demo}
shows a simple example.

\lstinputlisting[
  caption = Demonstration if the set\_bundled\_vertex\_my\_vertex function,
  label = lst:set_bundled_vertex_my_vertex_demo
]{set_my_bundled_vertex_demo.impl}

%%%%%%%%%%%%%%%%%%%%%%%%%%%%%%%%%%%%%%%%%%%%%%%%%%%%%%%%%%%%%%%%%%%%%%%%%%%%%%%%
\section{Setting all bundled vertices' my\_vertex objects}
\label{subsec:set_bundled_vertex_my_vertexes}
%%%%%%%%%%%%%%%%%%%%%%%%%%%%%%%%%%%%%%%%%%%%%%%%%%%%%%%%%%%%%%%%%%%%%%%%%%%%%%%%

When the vertices of a graph are \verb;my_bundled_vertex; objects, one can set
these as such:

\lstinputlisting[
  caption = Setting the bundled vertices' my\_bundled\_vertex-es,
  label = lst:set_bundled_vertex_my_vertexes
]{set_my_bundled_vertexes.impl}
\index{Set bundled vertex my\_bundled\_vertexes}

%%%%%%%%%%%%%%%%%%%%%%%%%%%%%%%%%%%%%%%%%%%%%%%%%%%%%%%%%%%%%%%%%%%%%%%%%%%%%%%%
\section{Storing a graph with bundled vertices as a .dot}
\label{subsec:save_bundled_vertices_graph_to_dot}
%%%%%%%%%%%%%%%%%%%%%%%%%%%%%%%%%%%%%%%%%%%%%%%%%%%%%%%%%%%%%%%%%%%%%%%%%%%%%%%%

If you used the \verb;create_bundled_vertices_k2_graph; function (algorithm 
\ref{lst:create_bundled_vertices_k2_graph}
) to produce a $K_{2}$ graph with vertices associated with 
\verb;my_bundled_vertex; objects, you can store these with algorithm 
\ref{lst:save_bundled_vertices_graph_to_dot}:

\lstinputlisting[
  caption = Storing a graph with bundled vertices as a .dot file,
  label = lst:save_bundled_vertices_graph_to_dot
]{save_bundled_vertices_graph_to_dot.impl}
\index{Save bundled vertices graph to dot}

This code looks small, because we call the \verb;make_bundled_vertices_writer;
function, which is shown in algorithm 
\ref{lst:make_bundled_vertices_writer}:

\lstinputlisting[
  caption = The make\_bundled\_vertices\_writer function,
  label = lst:make_bundled_vertices_writer
]{make_bundled_vertices_writer.impl}
\index{make\_bundled\_vertices\_writer}

Also this function is forwarding the real work to 
the \verb;bundled_vertices_writer;, shown in algorithm 
\ref{lst:bundled_vertices_writer}:

\lstinputlisting[
  caption = The bundled\_vertices\_writer function,
  label = lst:bundled_vertices_writer
]{bundled_vertices_writer.impl}
\index{bundled\_vertices\_writer}

Here, some interesting things are happening: the writer needs the bundled
property maps to work with and thus copies the whole graph to its internals.
I have chosen to map the \verb;my_bundled_vertex; member variables to Graphviz
attributes (see chapter \ref{subsec:Graphviz-attributes} 
for most Graphviz attributes) as shown in table 
\ref{tab:Mapping-of-my_bundled_vertex-to-Graphviz-attributes}:


\begin{table}[h!]
  \centering
  \begin{tabular}{l l l l} 
    \hline
    my\_bundled\_vertex variable & C++ data type & Graphviz data type & Graphviz attribute \\
    \hline
    m\_name                      & std::string   & string             & label              \\
    m\_description               & std::string   & string             & comment            \\
    m\_x                         & double        & double             & width              \\
    m\_y                         & double        & double             & height             \\
    \hline
  \end{tabular}
  \caption{
    Mapping of my\_bundled\_vertex member variable and Graphviz attributes
  }
  \label{tab:Mapping-of-my_bundled_vertex-to-Graphviz-attributes}
\end{table}

Important in this mapping is that the C++ and the Graphviz data types match.
I also chose attributes that matched as closely as possible.

The writer also encodes the std::string of the name and description to a
Graphviz-friendly format.
When loading the .dot file again, this will have to be undone again.

%%%%%%%%%%%%%%%%%%%%%%%%%%%%%%%%%%%%%%%%%%%%%%%%%%%%%%%%%%%%%%%%%%%%%%%%%%%%%%%%
\section{Loading a directed graph with bundled vertices from a .dot}
\label{subsec:load_directed_bundled_vertices_graph_from_dot}
%%%%%%%%%%%%%%%%%%%%%%%%%%%%%%%%%%%%%%%%%%%%%%%%%%%%%%%%%%%%%%%%%%%%%%%%%%%%%%%%

When loading a graph from file, one needs to specify a type of graph.
In this example, an directed graph with bundled vertices is loaded, as
shown in algorithm \ref{lst:load_directed_bundled_vertices_graph_from_dot}:

\lstinputlisting[
  caption = Loading a directed graph with bundled vertices from a .dot file,
  label = lst:load_directed_bundled_vertices_graph_from_dot
]{load_directed_bundled_vertices_graph_from_dot.impl}
\index{Load directed bundled vertices graph from dot}

In this algorithm, first it is checked if the file to load exists.
Then an empty directed graph is created, to save typing the typename explicitly.

Then a boost::dynamic_properties \index{boost::dynamic_properties}
is created with its default constructor, after which we set it to follow
the same mapping as in the previous chapter.
From this and the empty graph, \verb;boost::read_graphviz;
\index{boost::read_graphviz}
is called to build up the graph.

At the moment the graph is created, all \verb;my_bundled_vertex; their names
and description are in a Graphviz-friendly format.
By obtaining all vertex iterators and vertex descriptors, the encoding
is made undone.

Listing \ref{lst:load_directed_bundled_vertices_graph_from_dot_demo}
shows how to use the \verb;load_directed_bundled_vertices_graph_from_dot; function:

\lstinputlisting[
  caption = Demonstration of the load\_directed\_bundled\_vertices\_graph\_from\_dot function,
  label = lst:load_directed_bundled_vertices_graph_from_dot_demo
]{load_directed_bundled_vertices_graph_from_dot_demo.impl}

This demonstration shows how the Markov chain is created using 
the \verb;create_bundled_vertices_markov_chain; function 
(algorithm \ref{lst:create_bundled_vertices_markov_chain}), 
saved and then loaded.
The loaded graph is checked to be the same as the original.

%%%%%%%%%%%%%%%%%%%%%%%%%%%%%%%%%%%%%%%%%%%%%%%%%%%%%%%%%%%%%%%%%%%%%%%%%%%%%%%%
\section{Loading an undirected graph with bundled vertices from a .dot}
\label{subsec:load_undirected_bundled_vertices_graph_from_dot}
%%%%%%%%%%%%%%%%%%%%%%%%%%%%%%%%%%%%%%%%%%%%%%%%%%%%%%%%%%%%%%%%%%%%%%%%%%%%%%%%

When loading a graph from file, one needs to specify a type of graph.
In this example, an undirected graph with bundled vertices is loaded, as
shown in algorithm 
\ref{lst:load_undirected_bundled_vertices_graph_from_dot}:

\lstinputlisting[
  caption = Loading an undirected graph with bundled vertices from a .dot file,
  label = lst:load_undirected_bundled_vertices_graph_from_dot
]{load_undirected_bundled_vertices_graph_from_dot.impl}
\index{Load undirected bundled vertices graph from dot}

The only difference with loading a directed graph, is that the initial empty
graph is undirected instead.
Chapter \ref{subsec:load_directed_bundled_vertices_graph_from_dot}
describes the rationale of this function.

Listing \ref{lst:load_undirected_bundled_vertices_graph_from_dot_demo}
shows how to use the \verb;load_undirected_bundled_vertices_graph_from_dot;
function:

\lstinputlisting[
  caption = Demonstration of the load\_undirected\_bundled\_vertices\_graph\_from\_dot function,
  label = lst:\lload_undirected_bundled_vertices_graph_from_dot_demo
]{load_undirected_bundled_vertices_graph_from_dot_demo.impl}

This demonstration shows how $K_{2}$
with bundled vertices is created using the \verb;create_bundled_vertices_k2_graph;
function (algorithm \ref{lst:create_bundled_vertices_k2_graph}), 
saved and then loaded.
The loaded graph is checked to be the same as the original.

