%%%%%%%%%%%%%%%%%%%%%%%%%%%%%%%%%%%%%%%%%%%%%%%%%%%%%%%%%%%%%%%%%%%%%%%%%%%%%%%%
\chapter{Misc functions}
%%%%%%%%%%%%%%%%%%%%%%%%%%%%%%%%%%%%%%%%%%%%%%%%%%%%%%%%%%%%%%%%%%%%%%%%%%%%%%%%

These are some function I needed for creating this tutorial.
Although they are not important for working with graphs, I used these heavily.
These functions may be compiler-dependent, platform-dependent and/or there
may be superior alternatives.
I just add them for completeness.

%%%%%%%%%%%%%%%%%%%%%%%%%%%%%%%%%%%%%%%%%%%%%%%%%%%%%%%%%%%%%%%%%%%%%%%%%%%%%%%%
\section{Getting a data type as a std::string}
\label{subsec:get_type_name}
%%%%%%%%%%%%%%%%%%%%%%%%%%%%%%%%%%%%%%%%%%%%%%%%%%%%%%%%%%%%%%%%%%%%%%%%%%%%%%%%

This function will only work under GCC.
I found this code at 
\url{
  http://stackoverflow.com/questions/1055452/c-get-name-of-type-in-template
}.
Thanks to \verb;m-dudley' 
(Stack Overflow user page at \url{http://stackoverflow.com/users/111327/m-dudley}).

\lstinputlisting[
  caption = Getting a data type its name as a std::string,
  label = lst:get_type_name
]{get_type_name.impl}
\index{Get type name}

%%%%%%%%%%%%%%%%%%%%%%%%%%%%%%%%%%%%%%%%%%%%%%%%%%%%%%%%%%%%%%%%%%%%%%%%%%%%%%%%
\section{Convert a .dot to .svg}
\label{subsec:convert_dot_to_svg}
%%%%%%%%%%%%%%%%%%%%%%%%%%%%%%%%%%%%%%%%%%%%%%%%%%%%%%%%%%%%%%%%%%%%%%%%%%%%%%%%

All illustrations in this tutorial are created by converting .dot to a .svg
(\verb;Scalable Vector Graphic;) file.
This function assumes the program \verb;dot; is installed, which is part of
Graphviz.

\lstinputlisting[
  caption = Convert a .dot file to a .svg,
  label = lst:convert_dot_to_svg
]{convert_dot_to_svg.impl}
\index{Convert dot to svg}

\verb;convert_dot_to_svg; makes a system call 
to the program \verb;dot; to convert
the .dot file to an .svg file.

%%%%%%%%%%%%%%%%%%%%%%%%%%%%%%%%%%%%%%%%%%%%%%%%%%%%%%%%%%%%%%%%%%%%%%%%%%%%%%%%
\section{Check if a file exists}
\label{subsec:is_regular_file}
%%%%%%%%%%%%%%%%%%%%%%%%%%%%%%%%%%%%%%%%%%%%%%%%%%%%%%%%%%%%%%%%%%%%%%%%%%%%%%%%

Not the most smart way perhaps, but it does only use the STL.

\lstinputlisting[
  caption = Check if a file exists,
  label = lst:is_regular_file
]{is_regular_file.impl}
\index{Is regular file}

