%%%%%%%%%%%%%%%%%%%%%%%%%%%%%%%%%%%%%%%%%%%%%%%%%%%%%%%%%%%%%%%%%%%%%%%%%%%%%%%%
\chapter{Other graph functions}
\label{sec:Other-graph-functions}
%%%%%%%%%%%%%%%%%%%%%%%%%%%%%%%%%%%%%%%%%%%%%%%%%%%%%%%%%%%%%%%%%%%%%%%%%%%%%%%%

Some functions that did not fit in.

%%%%%%%%%%%%%%%%%%%%%%%%%%%%%%%%%%%%%%%%%%%%%%%%%%%%%%%%%%%%%%%%%%%%%%%%%%%%%%%%
\section{Encode a std::string to a Graphviz-friendly format}
\label{subsec:graphviz_encode}
%%%%%%%%%%%%%%%%%%%%%%%%%%%%%%%%%%%%%%%%%%%%%%%%%%%%%%%%%%%%%%%%%%%%%%%%%%%%%%%%

You may want to use a label with spaces, comma's and/or quotes.
Saving and loading these, will result in problem.
This function replaces these special characters by a rare combination of
ordinary characters.

\lstinputlisting[
  caption = Encode a std::string to a Graphviz-friendly format,
  label = lst:graphviz_encode
]{graphviz_encode.impl}
\index{graphviz encode}

%%%%%%%%%%%%%%%%%%%%%%%%%%%%%%%%%%%%%%%%%%%%%%%%%%%%%%%%%%%%%%%%%%%%%%%%%%%%%%%%
\section{Decode a std::string from a Graphviz-friendly format}
\label{subsec:graphviz_decode}
%%%%%%%%%%%%%%%%%%%%%%%%%%%%%%%%%%%%%%%%%%%%%%%%%%%%%%%%%%%%%%%%%%%%%%%%%%%%%%%%

This function undoes the \verb;graphviz_encode; function (algorithm 
\ref{lst:graphviz_encode}) and thus converts a 
Graphviz-friendly std::string to the original human-friendly std::string.

\lstinputlisting[
  caption = Decode a std::string from a Graphviz-friendly format to a human-friendly format,
  label = lst:graphviz_decode
]{graphviz_decode.impl}
\index{graphviz decode}

