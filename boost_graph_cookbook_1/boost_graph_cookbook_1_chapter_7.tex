%%%%%%%%%%%%%%%%%%%%%%%%%%%%%%%%%%%%%%%%%%%%%%%%%%%%%%%%%%%%%%%%%%%%%%%%%%%%%%%%
\chapter{Working on graphs with bundled edges and vertices}
%%%%%%%%%%%%%%%%%%%%%%%%%%%%%%%%%%%%%%%%%%%%%%%%%%%%%%%%%%%%%%%%%%%%%%%%%%%%%%%%

%%%%%%%%%%%%%%%%%%%%%%%%%%%%%%%%%%%%%%%%%%%%%%%%%%%%%%%%%%%%%%%%%%%%%%%%%%%%%%%%
\section{Has a my\_bundled\_edge}
\label{subsec:has_bundled_edge_with_my_edge}
%%%%%%%%%%%%%%%%%%%%%%%%%%%%%%%%%%%%%%%%%%%%%%%%%%%%%%%%%%%%%%%%%%%%%%%%%%%%%%%%

Before modifying our edges, let's first determine if we can find an edge
by its bundled type (\verb;my_bundled_edge;) in a graph.
After obtaining a \verb;my_bundled_edge; map, 
we obtain the edge iterators, dereference these to obtain 
the edge descriptors and then compare each edge its \verb;my_bundled_edge; 
with the one desired.

\lstinputlisting[
  caption = Find if there is a bundled edge with a certain my\_bundled\_edge,
  label = lst:has_bundled_edge_with_my_edge
]{has_bundled_edge_with_my_edge.impl}
\index{Has bundled edge with my\_bundled\_edge}

This function can be demonstrated as in algorithm 
\ref{lst:has_bundled_edge_with_my_edge_demo}, 
where a certain \verb;my_bundled_edge; cannot be found in an empty graph.
After adding the desired \verb;my_bundled_edge;, it is found.

\lstinputlisting[
  caption = Demonstration of the has\_bundled\_edge\_with\_my\_edge function,
  label = lst:has_bundled_edge_with_my_edge_demo
]{has_bundled_edge_with_my_edge_demo.impl}

Note that this function only finds if there is at least one edge with that
\verb;my_bundled_edge;: 
it does not tell how many edges with that \verb;my_bundled_edge;
exist in the graph.

%%%%%%%%%%%%%%%%%%%%%%%%%%%%%%%%%%%%%%%%%%%%%%%%%%%%%%%%%%%%%%%%%%%%%%%%%%%%%%%%
\section{Find a my\_bundled\_edge}
\label{subsec:find_first_bundled_edge_with_my_edge}
%%%%%%%%%%%%%%%%%%%%%%%%%%%%%%%%%%%%%%%%%%%%%%%%%%%%%%%%%%%%%%%%%%%%%%%%%%%%%%%%

Where STL functions work with iterators, here we obtain an edge descriptor
(see chapter \ref{subsec:Edge-descriptors}) 
to obtain a handle to the desired edge.
Listing \ref{lst:find_first_bundled_edge_with_my_edge}
shows how to obtain an edge descriptor to the first edge found with a specific
\verb;my_bundled_edge; value.

\lstinputlisting[
  caption = Find the first bundled edge with a certain my\_bundled\_edge,
  label = lst:find_first_bundled_edge_with_my_edge
]{find_first_bundled_edge_with_my_edge.impl}
\index{Find first bundled edge with my\_bundled\_edge}

With the edge descriptor obtained, one can read and modify the edge and
the vertices surrounding it.
Listing \ref{lst:find_first_bundled_edge_with_my_edge_demo}
shows some examples of how to do so.

\lstinputlisting[
  caption = Demonstration of the find\_first\_bundled\_edge\_with\_my\_edge function,
  label = lst:find_first_bundled_edge_with_my_edge_demo
]{find_first_bundled_edge_with_my_edge_demo.impl}

%%%%%%%%%%%%%%%%%%%%%%%%%%%%%%%%%%%%%%%%%%%%%%%%%%%%%%%%%%%%%%%%%%%%%%%%%%%%%%%%
\section{Get an edge its my\_bundled\_edge}
\label{subsec:get_bundled_edge_my_edge}
%%%%%%%%%%%%%%%%%%%%%%%%%%%%%%%%%%%%%%%%%%%%%%%%%%%%%%%%%%%%%%%%%%%%%%%%%%%%%%%%

To obtain the \verb;my_bundled_edge: from an edge descriptor, 
one needs to pull out the \verb;my_bundled_edges; map 
and then look up the \verb;my_edge; of interest.

\lstinputlisting[
  caption = Get a vertex its my\_bundled\_vertex from its vertex descriptor,
  label = lst:get_bundled_edge_my_edge
]{get_my_bundled_edge.impl}
\index{Get my\_bundled\_edge}

To use \verb;get_my_bundled_edge;, 
one first needs to obtain an edge descriptor.
Listing \ref{lst:get_bundled_edge_my_edge_demo} shows a simple example.

\lstinputlisting[
  caption = Demonstration if the get\_my\_bundled\_edge function,
  label = lst:get_bundled_edge_my_edge_demo
]{get_my_bundled_edge_demo.impl}

%%%%%%%%%%%%%%%%%%%%%%%%%%%%%%%%%%%%%%%%%%%%%%%%%%%%%%%%%%%%%%%%%%%%%%%%%%%%%%%%
\section{Set an edge its my\_bundled\_edge}
\label{subsec:set_bundled_edge_my_edge}
%%%%%%%%%%%%%%%%%%%%%%%%%%%%%%%%%%%%%%%%%%%%%%%%%%%%%%%%%%%%%%%%%%%%%%%%%%%%%%%%

If you know how to get the \verb;my_bundled_edge; 
from an edge descriptor, 
setting it is just as easy, 
as shown in algorithm \ref{lst:set_bundled_edge_my_edge}.

\lstinputlisting[
  caption = Set a bundled edge its my\_bundled\_edge from its edge descriptor,
  label = lst:set_bundled_edge_my_edge
]{set_my_bundled_edge.impl}
\index{Set bundled edge my\_bundled\_edge}

To use \verb;set_bundled_edge_my_edge;, 
one first needs to obtain an edge descriptor.
Listing \ref{lst:set_bundled_edge_my_edge_demo}
shows a simple example.

\lstinputlisting[
  caption = Demonstration if the set\_bundled\_edge\_my\_edge function,
  label = lst:set_bundled_edge_my_edge_demo
]{set_my_bundled_edge_demo.impl}

%%%%%%%%%%%%%%%%%%%%%%%%%%%%%%%%%%%%%%%%%%%%%%%%%%%%%%%%%%%%%%%%%%%%%%%%%%%%%%%%
\section{Storing a graph with bundled edges and vertices as a .dot}
\label{subsec:save_bundled_edges_and_vertices_graph_to_dot}
\index{Save graph with bundled edges and vertices as .dot}
\index{Create .dot from graph with bundled edges and vertices}
%%%%%%%%%%%%%%%%%%%%%%%%%%%%%%%%%%%%%%%%%%%%%%%%%%%%%%%%%%%%%%%%%%%%%%%%%%%%%%%%

If you used the \verb;create_bundled_edges_and_vertices_k3_graph; 
function (algorithm \ref{lst:create_bundled_edges_and_vertices_k3_graph}) 
to produce a $K_{3}$ graph with edges and vertices 
associated with \verb;my_bundled_edge; 
and \verb;my_bundled_vertex; objects, 
you can store these 
\verb;my_bundled_edges; and 
\verb;my_bundled_vertex;-es
additionally with algorithm \ref{lst:save_bundled_edges_and_vertices_graph_to_dot}:

\lstinputlisting[
  caption = Storing a graph with bundled edges and vertices as a .dot file,
  label = lst:save_bundled_edges_and_vertices_graph_to_dot
]{save_bundled_edges_and_vertices_graph_to_dot.impl}
\index{Save bundled edges and vertices graph to dot}

%%%%%%%%%%%%%%%%%%%%%%%%%%%%%%%%%%%%%%%%%%%%%%%%%%%%%%%%%%%%%%%%%%%%%%%%%%%%%%%%
\section{Load a directed graph with bundled edges and vertices from a .dot file}
\label{subsec:sub:load_directed_bundled_edges_and_vertices_graph_from_dot}
%%%%%%%%%%%%%%%%%%%%%%%%%%%%%%%%%%%%%%%%%%%%%%%%%%%%%%%%%%%%%%%%%%%%%%%%%%%%%%%%

When loading a graph from file, one needs to specify a type of graph.
In this example, an directed graph with bundled edges and vertices is loaded,
as shown in algorithm 
\ref{lst:load_directed_bundled_edges_and_vertices_graph_from_dot}:

\lstinputlisting[
  caption = Loading a directed graph with bundled edges and vertices from a .dot file,
  label = lst:load_directed_bundled_edges_and_vertices_graph_from_dot
]{load_directed_bundled_edges_and_vertices_graph_from_dot.impl}
\index{Load directed bundled edges and vertices graph from dot}

In this algorithm, first it is checked if the file to load exists.
Then an empty directed graph is created.
Next to this, a 
\verb;boost::dynamic_properties; \index{boost::dynamic\_properties}
is created with its default constructor, after which we direct the 
\verb;boost::dynamic_properties; \index{boost::dynamic\_properties}
to find a \verb;node_id; and \verb;label; in the vertex name map, 
\verb;edge_id; and \verb;label; to the edge name map.
 From this and the empty graph, 
\verb;boost::read_graphviz; \index{boost::read\_graphviz}
is called to build up the graph.

Listing 
\ref{lst:load_directed_bundled_edges_and_vertices_graph_from_dot_demo}
shows how to use the \verb;load_directed_bundled_edges_and_vertices_graph_from_dot;
function:

\lstinputlisting[
  caption = Demonstration of the load\_directed\_bundled\_edges\_and\_vertices\_graph\_from\_dot function,
  label = lst:load_directed_bundled_edges_and_vertices_graph_from_dot_demo
]{load_directed_bundled_edges_and_vertices_graph_from_dot_demo.impl}

This demonstration shows how the Markov chain is created using 
the \verb;create_bundled_edges_and_vertices_markov_chain; function 
(algorithm \ref{lst:create_bundled_edges_and_vertices_markov_chain}), 
saved and then loaded.

%%%%%%%%%%%%%%%%%%%%%%%%%%%%%%%%%%%%%%%%%%%%%%%%%%%%%%%%%%%%%%%%%%%%%%%%%%%%%%%%
\section{Load an undirected graph with bundled edges and vertices from a .dot file}
\label{subsec:load_undirected_bundled_edges_and_vertices_graph_from_dot}
\index{Load undirected graph with bundled edges and vertices from .dot}
\index{Create undirected graph with bundled edges and vertices from .dot}
%%%%%%%%%%%%%%%%%%%%%%%%%%%%%%%%%%%%%%%%%%%%%%%%%%%%%%%%%%%%%%%%%%%%%%%%%%%%%%%%

When loading a graph from file, one needs to specify a type of graph.
In this example, an undirected graph with bundled edges and vertices is
loaded, as shown in algorithm 
\ref{lst:load_undirected_bundled_edges_and_vertices_graph_from_dot}:

\lstinputlisting[
  caption = Loading an undirected graph with bundled edges and vertices from a .dot file,
  label = lst:load_undirected_bundled_edges_and_vertices_graph_from_dot
]{load_undirected_bundled_edges_and_vertices_graph_from_dot.impl}
\index{Load undirected bundled edges and vertices graph from dot}

The only difference with loading a directed graph, is that the initial empty
graph is undirected instead.

Chapter \ref{subsec:sub:load_directed_bundled_edges_and_vertices_graph_from_dot}
describes the rationale of this function.

Listing \ref{lst:load_undirected_bundled_edges_and_vertices_graph_from_dot_demo}
shows how to use the \verb;load_undirected_bundled_vertices_graph_from_dot;
function:

\lstinputlisting[
  caption = Demonstration of the load\_undirected\_bundled\_edges\_and\_vertices\_graph\_from\_dot function,
  label = lst:load_undirected_bundled_edges_and_vertices_graph_from_dot_demo
]{load_undirected_bundled_edges_and_vertices_graph_from_dot_demo.impl}

This demonstration shows how $K_{2}$
with bundled vertices is created using the 
\verb;create_bundled_vertices_k2_graph; function 
(algorithm \ref{lst:create_bundled_vertices_k2_graph}), 
saved and then loaded. 
The loaded graph is checked to be a graph similar to the original.

