%%%%%%%%%%%%%%%%%%%%%%%%%%%%%%%%%%%%%%%%%%%%%%%%%%%%%%%%%%%%%%%%%%%%%%%%%%%%%%%%
\chapter{Working on graphs with bundled edges and vertices}
%%%%%%%%%%%%%%%%%%%%%%%%%%%%%%%%%%%%%%%%%%%%%%%%%%%%%%%%%%%%%%%%%%%%%%%%%%%%%%%%

%%%%%%%%%%%%%%%%%%%%%%%%%%%%%%%%%%%%%%%%%%%%%%%%%%%%%%%%%%%%%%%%%%%%%%%%%%%%%%%%
\subsection{Has a my\_bundled\_edge}
\label{subsec:has_bundled_edge_with_my_edge}
%%%%%%%%%%%%%%%%%%%%%%%%%%%%%%%%%%%%%%%%%%%%%%%%%%%%%%%%%%%%%%%%%%%%%%%%%%%%%%%%

Before modifying our edges, let's first determine if we can find an edge
by its bundled type (\verb;my_bundled_edge;) in a graph.
After obtaining a my\_bundled\_edge map, 
we obtain the edge iterators, dereference these to obtain 
the edge descriptors and then compare each edge its my\_bundled\_edge 
with the one desired.

\lstinputlisting[
  caption = Find if there is a bundled edge with a certain my\_bundled\_edge,
  label = lst:has_bundled_edge_with_my_edge
]{has_bundled_edge_with_my_edge.impl}
\index{Has bundled edge with my_bundled_edge}

This function can be demonstrated as in algorithm 
\ref{lst:has_bundled_edge_with_my_edge_demo}, 
where a certain \verb;my_bundled_edge; cannot be found in an empty graph.
After adding the desired my\_bundled\_edge, it is found.

\lstinputlisting[
  caption = Demonstration of the has\_bundled\_edge\_with\_my\_edge function,
  label = lst:has_bundled_edge_with_my_edge_demo
]{has_bundled_edge_with_my_edge_demo.impl}

Note that this function only finds if there is at least one edge with that
my\_bundled\_edge: it does not tell how many edges with that my_bundled_edge
exist in the graph.

