%%%%%%%%%%%%%%%%%%%%%%%%%%%%%%%%%%%%%%%%%%%%%%%%%%%%%%%%%%%%%%%%%%%%%%%%%%%%%%%%
\chapter{Working on graphs with bundled edges and vertices}
%%%%%%%%%%%%%%%%%%%%%%%%%%%%%%%%%%%%%%%%%%%%%%%%%%%%%%%%%%%%%%%%%%%%%%%%%%%%%%%%

%%%%%%%%%%%%%%%%%%%%%%%%%%%%%%%%%%%%%%%%%%%%%%%%%%%%%%%%%%%%%%%%%%%%%%%%%%%%%%%%
\section{Has a my\_bundled\_edge}
\label{subsec:has_bundled_edge_with_my_edge}
%%%%%%%%%%%%%%%%%%%%%%%%%%%%%%%%%%%%%%%%%%%%%%%%%%%%%%%%%%%%%%%%%%%%%%%%%%%%%%%%

Before modifying our edges, let's first determine if we can find an edge
by its bundled type (\verb;my_bundled_edge;) in a graph.
After obtaining a my\_bundled\_edge map, 
we obtain the edge iterators, dereference these to obtain 
the edge descriptors and then compare each edge its my\_bundled\_edge 
with the one desired.

\lstinputlisting[
  caption = Find if there is a bundled edge with a certain my\_bundled\_edge,
  label = lst:has_bundled_edge_with_my_edge
]{has_bundled_edge_with_my_edge.impl}
\index{Has bundled edge with my_bundled_edge}

This function can be demonstrated as in algorithm 
\ref{lst:has_bundled_edge_with_my_edge_demo}, 
where a certain \verb;my_bundled_edge; cannot be found in an empty graph.
After adding the desired my\_bundled\_edge, it is found.

\lstinputlisting[
  caption = Demonstration of the has\_bundled\_edge\_with\_my\_edge function,
  label = lst:has_bundled_edge_with_my_edge_demo
]{has_bundled_edge_with_my_edge_demo.impl}

Note that this function only finds if there is at least one edge with that
my\_bundled\_edge: it does not tell how many edges with that my_bundled_edge
exist in the graph.

%%%%%%%%%%%%%%%%%%%%%%%%%%%%%%%%%%%%%%%%%%%%%%%%%%%%%%%%%%%%%%%%%%%%%%%%%%%%%%%%
\section{Find a my\_bundled\_edge}
\label{subsec:find_first_bundled_edge_with_my_edge}
%%%%%%%%%%%%%%%%%%%%%%%%%%%%%%%%%%%%%%%%%%%%%%%%%%%%%%%%%%%%%%%%%%%%%%%%%%%%%%%%

Where STL functions work with iterators, here we obtain an edge descriptor
(see chapter \ref{subsec:Edge-descriptors}) 
to obtain a handle to the desired edge.
Listing \ref{lst:find_first_bundled_edge_with_my_edge}
shows how to obtain an edge descriptor to the first edge found with a specific
my_bundled_edge value.

\lstinputlisting[
  caption = Find the first bundled edge with a certain my\_bundled\_edge,
  label = lst:find_first_bundled_edge_with_my_edge
]{find_first_bundled_edge_with_my_edge.impl}
\index{Find first bundled edge with my_bundled_edge}

With the edge descriptor obtained, one can read and modify the edge and
the vertices surrounding it.
Listing \ref{lst:find_first_bundled_edge_with_my_edge_demo}
shows some examples of how to do so.

\lstinputlisting[
  caption = Demonstration of the find\_first\_bundled\_edge\_with\_my\_edge function,
  label = lst:find_first_bundled_edge_with_my_edge_demo
]{find_first_bundled_edge_with_my_edge_demo.impl}

%%%%%%%%%%%%%%%%%%%%%%%%%%%%%%%%%%%%%%%%%%%%%%%%%%%%%%%%%%%%%%%%%%%%%%%%%%%%%%%%
\section{Get an edge its my\_bundled\_edge}
\label{subsec:get_bundled_edge_my_edge}
%%%%%%%%%%%%%%%%%%%%%%%%%%%%%%%%%%%%%%%%%%%%%%%%%%%%%%%%%%%%%%%%%%%%%%%%%%%%%%%%

To obtain the my\_bundled\_edge from an edge descriptor, one needs to pull
out the my\_bundled_edges map and then look up the my\_edge of interest.

\lstinputlisting[
  caption = Get a vertex its my\_bundled\_vertex from its vertex descriptor,
  label = lst:get_bundled_edge_my_edge
]{get_my_bundled_edge.impl}
\index{Get my_bundled_edge}

To use \verb;get_my_bundled_edge;, 
one first needs to obtain an edge descriptor.
Listing \ref{lst:get_bundled_edge_my_edge_demo} shows a simple example.

\lstinputlisting[
  caption = Demonstration if the get\_my\_bundled\_edge function,
  label = lst:get_bundled_edge_my_edge_demo
]{get_my_bundled_edge_demo.impl}

