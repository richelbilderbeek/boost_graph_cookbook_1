%%%%%%%%%%%%%%%%%%%%%%%%%%%%%%%%%%%%%%%%%%%%%%%%%%%%%%%%%%%%%%%%%%%%%%%%%%%%%%%%
\chapter{Building graphs without properties}
\label{sec:Building-graphs-without-properties}
%%%%%%%%%%%%%%%%%%%%%%%%%%%%%%%%%%%%%%%%%%%%%%%%%%%%%%%%%%%%%%%%%%%%%%%%%%%%%%%%

Boost.Graph is about creating graphs.
In this chapter we create the simplest of graphs, in which edges and nodes
have no properties (e.g. having a name).

Still, there are two types of graphs that can be constructed: undirected
and directed graphs.
The difference between directed and undirected graphs is in the edges:
in an undirected graph, 
an edge connects two vertices without any directionality, as displayed
in figure \ref{fig:undirected_graph_example}.

In a directed graph, an edge goes from a certain vertex, its source, to another (which may
 actually be the same), its target.
 A directed graph is shown in figure 
\ref{fig:directed_graph_example}.
