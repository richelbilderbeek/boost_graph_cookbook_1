%%%%%%%%%%%%%%%%%%%%%%%%%%%%%%%%%%%%%%%%%%%%%%%%%%%%%%%%%%%%%%%%%%%%%%%%%%%%%%%%
\chapter{Building graphs without properties}
\label{sec:Building-graphs-without-properties}
%%%%%%%%%%%%%%%%%%%%%%%%%%%%%%%%%%%%%%%%%%%%%%%%%%%%%%%%%%%%%%%%%%%%%%%%%%%%%%%%

Boost.Graph is about creating graphs.
In this chapter we create the simplest of graphs, in which edges and nodes
have no properties (e.g. having a name).

Still, there are two types of graphs that can be constructed: undirected
and directed graphs.
The difference between directed and undirected graphs is in the edges:
in an undirected graph, 
an edge connects two vertices without any directionality, as displayed
in figure \ref{fig:undirected_graph_example}.

In a directed graph, an edge goes from a certain vertex, 
its source, to another (which may actually be the same), its target.
A directed graph is shown in figure \ref{fig:directed_graph_example}.

\begin{figure}
  \begin{tikzpicture}
    \draw[thick] 
      (0,0) node[draw=black,fill=white,shape=circle,text=white] {} 
      -- (5,2) node[draw=black,fill=white,shape=circle,text=white] {} 
      -- (10,1) node[draw=black,fill=white,shape=circle,text=white] {} 
    ;
  \end{tikzpicture}
  \caption{Example of an undirected graph}
  \label{fig:undirected_graph_example}
\end{figure}

\begin{figure}
  \begin{tikzpicture}
    \SetGraphUnit{5}
    \tikzset{VertexStyle/.append style = {draw=black,fill=white,shape=circle,text=white} }
    \Vertex{A}
    \EA(A){B}
    \EA(B){C}
    \tikzset{EdgeStyle/.append style = {->, bend left} }
    \Edge[](A)(B)
    \Edge[](B)(A)
    \Loop[dist = 4cm, dir = NO](A.west)
    \tikzset{EdgeStyle/.append style = {bend left = 0} }
    \Edge[](C)(B)   
  \end{tikzpicture}
  \caption{Example of a directed graph}
  \label{fig:directed_graph_example}
\end{figure}

