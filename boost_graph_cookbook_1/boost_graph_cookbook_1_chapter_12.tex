%%%%%%%%%%%%%%%%%%%%%%%%%%%%%%%%%%%%%%%%%%%%%%%%%%%%%%%%%%%%%%%%%%%%%%%%%%%%%%%%
\chapter{Errors}
%%%%%%%%%%%%%%%%%%%%%%%%%%%%%%%%%%%%%%%%%%%%%%%%%%%%%%%%%%%%%%%%%%%%%%%%%%%%%%%%

Some common errors.

%%%%%%%%%%%%%%%%%%%%%%%%%%%%%%%%%%%%%%%%%%%%%%%%%%%%%%%%%%%%%%%%%%%%%%%%%%%%%%%%
\section{Formed reference to void}
\label{subsec:formed_reference_to_void}
%%%%%%%%%%%%%%%%%%%%%%%%%%%%%%%%%%%%%%%%%%%%%%%%%%%%%%%%%%%%%%%%%%%%%%%%%%%%%%%%

This compile-time error occurs when you create a graph without a certain
 property, then subsequently reading that property, as in algorithm 
\ref{lst:formed_reference_to_void}: 

\lstinputlisting[
  caption = Creating the error 'formed reference to void',
  label = lst:formed_reference_to_void
]{formed_reference_to_void.impl}
\index{Formed reference to void}

In algorithm \ref{lst:formed_reference_to_void}
a graph is created with vertices of no properties.
Then the names of these vertices, which do not exists, are tried to be
read.
If you want to read the names of the vertices, supply a graph that has
this property.

%%%%%%%%%%%%%%%%%%%%%%%%%%%%%%%%%%%%%%%%%%%%%%%%%%%%%%%%%%%%%%%%%%%%%%%%%%%%%%%%
\section{No matching function for call to clear\_out\_edges}
\label{subsec:no_matching_function_for_call_to_clear_out_edges}
%%%%%%%%%%%%%%%%%%%%%%%%%%%%%%%%%%%%%%%%%%%%%%%%%%%%%%%%%%%%%%%%%%%%%%%%%%%%%%%%

This compile-time error occurs when you want to clear the outward edges
from a vertex in an undirected graph.

\lstinputlisting[
  caption = Creating the error 'no matching function for call to clear\_out\_edges',
  label = lst:no_matching_function_for_call_to_clear_out_edges
]{no_matching_function_for_call_to_clear_out_edges.impl}
\index{No matching function for call to clear\_out\_edges}

In algorithm 
\ref{lst:no_matching_function_for_call_to_clear_out_edges}
an undirected graph is created, a vertex descriptor is obtained, then its
out
edges are tried to be cleared. Either use a directed graph (which has out
edges), or use the \verb;boost::clear_vertex; function instead.

%%%%%%%%%%%%%%%%%%%%%%%%%%%%%%%%%%%%%%%%%%%%%%%%%%%%%%%%%%%%%%%%%%%%%%%%%%%%%%%%
\section{No matching function for call to \verb;clear_in_edges;}
\label{subsec:no_matching_function_for_call_to_clear_in_edges}
%%%%%%%%%%%%%%%%%%%%%%%%%%%%%%%%%%%%%%%%%%%%%%%%%%%%%%%%%%%%%%%%%%%%%%%%%%%%%%%%

See chapter \ref{subsec:no_matching_function_for_call_to_clear_out_edges}.

%%%%%%%%%%%%%%%%%%%%%%%%%%%%%%%%%%%%%%%%%%%%%%%%%%%%%%%%%%%%%%%%%%%%%%%%%%%%%%%%
\section{Undefined reference to boost::detail::graph::read_graphviz_new}
\label{subsec:undefined_reference_to_read_graphviz_new}
\index{read\_graphviz\_new}
\index{Undefined reference to read_graphviz\_new}
\index{read\_graphviz\_new, undefined reference}
%%%%%%%%%%%%%%%%%%%%%%%%%%%%%%%%%%%%%%%%%%%%%%%%%%%%%%%%%%%%%%%%%%%%%%%%%%%%%%%%

You will have to link \index{link}
against the Boost.Graph and Boost.Regex libraries.
In Qt Creator, this is achieved by adding these lines to your Qt Creator
project file:

\begin{verbatim}
LIBS += -lboost_graph -lboost_regex 
\end{verbatim}

