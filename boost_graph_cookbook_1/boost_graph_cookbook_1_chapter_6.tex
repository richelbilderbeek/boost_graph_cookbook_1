%%%%%%%%%%%%%%%%%%%%%%%%%%%%%%%%%%%%%%%%%%%%%%%%%%%%%%%%%%%%%%%%%%%%%%%%%%%%%%%%
\chapter{Building graphs with bundled edges and vertices}
%%%%%%%%%%%%%%%%%%%%%%%%%%%%%%%%%%%%%%%%%%%%%%%%%%%%%%%%%%%%%%%%%%%%%%%%%%%%%%%%

Up until now, the graphs created have had only bundled vertices.
In this chapter, graphs will be created, in which both the edges and vertices
have a bundled \verb;my_bundled_edge; and \verb;my_bundled_edge; type
\footnote{I do not intend to be original in naming my data types}.

\begin{itemize}
  \item An empty directed graph that allows for bundled edges and vertices: 
    see chapter \ref{subsec:create_empty_directed_bundled_edges_and_vertices_graph}
  \item An empty undirected graph that allows for bundled edges and vertices: 
    see chapter \ref{subsec:create_empty_undirected_bundled_edges_and_vertices_graph}
  \item A two-state Markov chain with bundled edges and vertices: 
    see chapter \ref{subsec:create_bundled_edges_and_vertices_markov_chain}
  \item $K_{3}$ with bundled edges and vertices: 
    see chapter \ref{subsec:create_bundled_edges_and_vertices_k3}
\end{itemize}

In the process, some basic (sometimes bordering trivial) functions are shown:

\begin{itemize}
  \item Creating the \verb;my_bundled_edge; class: 
    see chapter \ref{subsec:my_bundled_edge}
  \item Adding a bundled \verb;my_bundled_edge;: 
    see chapter \ref{subsec:add_bundled_edge}
\end{itemize}

These functions are mostly there for completion and showing which data types
are used.

%%%%%%%%%%%%%%%%%%%%%%%%%%%%%%%%%%%%%%%%%%%%%%%%%%%%%%%%%%%%%%%%%%%%%%%%%%%%%%%%
\section{Creating the bundled edge class}
\label{subsec:my_bundled_edge}
%%%%%%%%%%%%%%%%%%%%%%%%%%%%%%%%%%%%%%%%%%%%%%%%%%%%%%%%%%%%%%%%%%%%%%%%%%%%%%%%

In this example, I create a \verb;my_bundled_edge; class.
Here I will show the header file of it, as the implementation of it is
not important yet.

\lstinputlisting[
  caption = Declaration of my\_bundled\_edge,
  label = lst:my_bundled_edge_h
]{my_bundled_edge.impl}
\index{my\_bundled\_edge}
\index{my\_bundled\_edge.h}
\index{my\_bundled\_edge declaration}
\index{Declaration, my\_bundled\_edge}

my\_bundled\_edge is a class that has multiple properties: 
two doubles \verb;m_width; 
(\verb;m_; \index{m\_} stands for member \index{member}) 
and \verb;m_height;, 
and two std::strings \verb;m_name; and \verb;m_description;.
\verb;my_bundled_edge; is copyable, 
but cannot trivially be converted to a \verb;std::string;. 
\verb;my_bundled_edge; is comparable for equality 
(that is, operator== is defined).
\verb;my_bundled_edge; does not have to have the stream operators defined for
file I/O, as this goes via the public member variables.

