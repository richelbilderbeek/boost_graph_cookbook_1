%%%%%%%%%%%%%%%%%%%%%%%%%%%%%%%%%%%%%%%%%%%%%%%%%%%%%%%%%%%%%%%%%%%%%%%%%%%%%%%%
\chapter{Building graphs with bundled edges and vertices}
%%%%%%%%%%%%%%%%%%%%%%%%%%%%%%%%%%%%%%%%%%%%%%%%%%%%%%%%%%%%%%%%%%%%%%%%%%%%%%%%

Up until now, the graphs created have had only bundled vertices.
In this chapter, graphs will be created, in which both the edges and vertices
have a bundled \verb;my_bundled_edge; and \verb;my_bundled_edge; type
\footnote{I do not intend to be original in naming my data types}.

\begin{itemize}
  \item An empty directed graph that allows for bundled edges and vertices: 
    see chapter \ref{subsec:create_empty_directed_bundled_edges_and_vertices_graph}
  \item An empty undirected graph that allows for bundled edges and vertices: 
    see chapter \ref{subsec:create_empty_undirected_bundled_edges_and_vertices_graph}
  \item A two-state Markov chain with bundled edges and vertices: 
    see chapter \ref{subsec:create_bundled_edges_and_vertices_markov_chain}
  \item $K_{3}$ with bundled edges and vertices: 
    see chapter \ref{subsec:create_bundled_edges_and_vertices_k3}
\end{itemize}

In the process, some basic (sometimes bordering trivial) functions are shown:

\begin{itemize}
  \item Creating the \verb;my_bundled_edge; class: 
    see chapter \ref{subsec:my_bundled_edge}
  \item Adding a bundled \verb;my_bundled_edge;: 
    see chapter \ref{subsec:add_bundled_edge}
\end{itemize}

These functions are mostly there for completion and showing which data types
are used.

