%%%%%%%%%%%%%%%%%%%%%%%%%%%%%%%%%%%%%%%%%%%%%%%%%%%%%%%%%%%%%%%%%%%%%%%%%%%%%%%%
\chapter{Working on graphs without properties}
\label{sec:Working-on-graphs-without-properties}
%%%%%%%%%%%%%%%%%%%%%%%%%%%%%%%%%%%%%%%%%%%%%%%%%%%%%%%%%%%%%%%%%%%%%%%%%%%%%%%%

Now that we can build a graph, there are some things we can do.

\begin{itemize}
  \item
    Getting the vertices' out degrees: 
    see chapter \ref{subsec:get_vertex_out_degrees}
  \item
    Create a direct-neighbour subgraph from a vertex descriptor
  \item
    Create all direct-neighbour subgraphs from a graphs
  \item
    Saving a graph without properties to .dot file: 
    see chapter \ref{subsec:save_graph_to_dot}
  \item
    Loading an undirected graph without properties from .dot file: 
    see chapter \ref{subsec:load_undirected_graph_from_dot}
  \item
    Loading a directed graph without properties from .dot file: 
    see chapter \ref{subsec:load_directed_graph_from_dot}
\end{itemize}


%%%%%%%%%%%%%%%%%%%%%%%%%%%%%%%%%%%%%%%%%%%%%%%%%%%%%%%%%%%%%%%%%%%%%%%%%%%%%%%%
\section{Getting the vertices' out degree}
\label{subsec:get_vertex_out_degrees}
%%%%%%%%%%%%%%%%%%%%%%%%%%%%%%%%%%%%%%%%%%%%%%%%%%%%%%%%%%%%%%%%%%%%%%%%%%%%%%%%

Let's measure the out degree of all vertices in a graph! 

The out degree of a vertex is the number of edges that originate at it.

The number of connections is called the \verb;degree; of the vertex.
There are three types of degrees:

\begin{itemize}
  \item in degree: the number of incoming connections, 
    using \verb;in_degree; \index{in\_degree}
    (not \verb;boost::in_degree; \index{boost::in\_degree does not exist} )
  \item out degree: the number of outgoing connections, 
    using \verb;out_degree; \index{out\_degree}
    (not \verb;boost::out_degree; \index{boost::out\_degree does not exist})
  \item degree: sum of the in degree and out degree, 
    using \verb;degree; \index{degree}
    (not \verb;boost::degree; \index{boost::degree does not exist})
\end{itemize}

Listing \ref{lst:get_vertex_out_degrees}
shows how to obtain these:

\lstinputlisting[
  caption = Get the vertices' out degrees,
  label = lst:get_vertex_out_degrees
]{get_vertex_out_degrees.impl}
\index{Get vertex out degrees}

The structure of this algorithm is similar to 
\verb;get_vertex_descriptors; (algorithm \ref{lst:get_vertex_descriptors}), 
except that the out degrees from the vertex descriptors are stored.
The out degree of a vertex iterator is obtained from 
the function \verb;out_degree; \index{out\_degree}
(not boost::out_degree \index{boost::out\_degree does not exist}!).
 
Albeit that the $K_{2}$
graph and the two-state Markov chain are rather simple, we can use it to
demonstrate \verb;get_vertex_out_degrees; on, as shown in algorithm 
\ref{lst:get_vertex_out_degrees_demo}.

\lstinputlisting[
  caption = Demonstration of the get\_vertex\_out\_degrees function,
  label = lst:get_vertex_out_degrees_demo
]{get_vertex_out_degrees_demo.impl}

It is expected that $K_{2}$
has one out-degree for every vertex, where the two-state Markov chain is
expected to have two out-degrees per vertex.

