%%%%%%%%%%%%%%%%%%%%%%%%%%%%%%%%%%%%%%%%%%%%%%%%%%%%%%%%%%%%%%%%%%%%%%%%%%%%%%%%
\chapter{Building graphs with bundled vertices}
\label{sec:Building-graphs-with-bundled-vertices}
%%%%%%%%%%%%%%%%%%%%%%%%%%%%%%%%%%%%%%%%%%%%%%%%%%%%%%%%%%%%%%%%%%%%%%%%%%%%%%%%

Up until now, the graphs created have had edges and vertices without any
properties.
In this chapter, graphs will be created, in which the vertices can have
a bundled \verb;my_bundled_vertex; type
\footnote{I do not intend to be original in naming my data types}.
The following graphs will be created:

\begin{itemize}
  \item An empty directed graph that allows for bundled vertices: 
    see chapter \ref{lst:create_empty_directed_bundled_vertices_graph}
  \item An empty undirected graph that allows for bundled vertices: 
    see chapter \ref{subsec:create_empty_directed_bundled_vertices_graph}
  \item A two-state Markov chain with bundled vertices: 
    see chapter \ref{subsec:create_bundled_vertices_markov_chain}
  \item $K_{2}$ with bundled vertices: 
    see chapter \ref{subsec:create_bundled_vertices_k2_graph}
\end{itemize}

In the process, some basic (sometimes bordering trivial) functions are shown:

\begin{itemize}
  \item 
    Create the vertex class, called \verb;my_bundled_vertex;: 
    see chapter \ref{subsec:my_bundled_vertex}
  \item Adding a \verb;my_bundled_vertex;: 
    see chapter \ref{subsec:add_bundled_vertex}
  \item Getting the vertices \verb;my_bundled_vertex;-es: 
    see chapter \ref{subsec:get_bundled_vertex_my_vertexes}
\end{itemize}

These functions are mostly there for completion and showing which data types
are used.

