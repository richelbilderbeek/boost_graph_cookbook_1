%%%%%%%%%%%%%%%%%%%%%%%%%%%%%%%%%%%%%%%%%%%%%%%%%%%%%%%%%%%%%%%%%%%%%%%%%%%%%%%%
\chapter{Building graphs with bundled vertices}
\label{sec:Building-graphs-with-bundled-vertices}
%%%%%%%%%%%%%%%%%%%%%%%%%%%%%%%%%%%%%%%%%%%%%%%%%%%%%%%%%%%%%%%%%%%%%%%%%%%%%%%%

Up until now, the graphs created have had edges and vertices without any
properties.
In this chapter, graphs will be created, in which the vertices can have
a bundled \verb;my_bundled_vertex; type
\footnote{I do not intend to be original in naming my data types}.
The following graphs will be created:

\begin{itemize}
  \item An empty directed graph that allows for bundled vertices: 
    see chapter \ref{lst:create_empty_directed_bundled_vertices_graph}
  \item An empty undirected graph that allows for bundled vertices: 
    see chapter \ref{subsec:create_empty_directed_bundled_vertices_graph}
  \item A two-state Markov chain with bundled vertices: 
    see chapter \ref{subsec:create_bundled_vertices_markov_chain}
  \item $K_{2}$ with bundled vertices: 
    see chapter \ref{subsec:create_bundled_vertices_k2_graph}
\end{itemize}

In the process, some basic (sometimes bordering trivial) functions are shown:

\begin{itemize}
  \item 
    Create the vertex class, called \verb;my_bundled_vertex;: 
    see chapter \ref{subsec:my_bundled_vertex}
  \item Adding a \verb;my_bundled_vertex;: 
    see chapter \ref{subsec:add_bundled_vertex}
  \item Getting the vertices \verb;my_bundled_vertex;-es: 
    see chapter \ref{subsec:get_bundled_vertex_my_vertexes}
\end{itemize}

These functions are mostly there for completion and showing which data types
are used.

%%%%%%%%%%%%%%%%%%%%%%%%%%%%%%%%%%%%%%%%%%%%%%%%%%%%%%%%%%%%%%%%%%%%%%%%%%%%%%%%
\section{Creating the bundled vertex class}
\label{subsec:my_bundled_vertex}
%%%%%%%%%%%%%%%%%%%%%%%%%%%%%%%%%%%%%%%%%%%%%%%%%%%%%%%%%%%%%%%%%%%%%%%%%%%%%%%%

Before creating an empty graph with bundled vertices, that bundled vertex
class must be created.
In this tutorial, it is called \verb;my_bundled_vertex;.
\verb;my_bundled_vertex; is a class that is nonsensical, but it can be replaced
by any other class type.

Here I will show the header file of \verb;my_bundled_vertex;, as the implementation
of it is not important:

\lstinputlisting[
  caption = Declaration of my\_bundled\_vertex,
  label = lst:my_bundled_vertex_h
]{my_bundled_vertex.impl}
\index{my_bundled_vertex}
\index{my_bundled_vertex.h}
\index{my_vertex declaration}
\index{Declaration, my_bundled_vertex}

\verb;my_bundled_vertex; is a class that has multiple properties: 

\begin{itemize}
  \item
    It has four public member variables: 
    the double \verb;m_x; 
    (\verb;m_; \index{m\_} stands for 'member' \index{member}), 
    the double \verb;m_y;, 
    the \verb;std::string; \verb;m_name; and the 
    \verb;std::string; \verb;m_description;.
    These variables must be public
  \item It has a default constructor
  \item It is copyable
  \item It is comparable for equality 
    (it has \verb;operator==;), 
    which is needed for searching
\end{itemize}

\verb;my_bundled_vertex; does not have to have the stream operators defined for
file I/O, as this goes via the public member variables.

%%%%%%%%%%%%%%%%%%%%%%%%%%%%%%%%%%%%%%%%%%%%%%%%%%%%%%%%%%%%%%%%%%%%%%%%%%%%%%%%
\section{Create the empty directed graph with bundled vertices}
\label{subsec:create_empty_directed_bundled_vertices_graph}
%%%%%%%%%%%%%%%%%%%%%%%%%%%%%%%%%%%%%%%%%%%%%%%%%%%%%%%%%%%%%%%%%%%%%%%%%%%%%%%%

\lstinputlisting[
  caption = Creating an empty directed graph with bundled vertices,
  label = lst:create_empty_directed_bundled_vertices_graph
]{create_empty_directed_bundled_vertices_graph.impl}
\index{Create empty directed bundled vertices graph}

This graph:

\begin{itemize}
  \item has its out edges stored in a std::vector 
    (due to the first boost::vecS \index{boost::vecS})
  \item has its vertices stored in a std::vector 
    (due to the second boost::vecS \index{boost::vecS})
  \item is directed 
    (due to the boost::directedS \index{boost::directedS})
  \item The vertices have one property: 
    they have a bundled type, 
    that is of data type \verb;my_bundled_vertex;
  \item The edges and graph have no properties
  \item Edges are stored in a std::list
\end{itemize}

The \verb;boost::adjacency_list; has a new, fourth template argument 
\verb;my_bundled_vertex; \index{my_bundled_vertex}.

This can be read as: 

\begin{quote}
vertices have the bundled property \verb;my_bundled_vertex;
\end{quote}

Or simply: 

\begin{quote}
vertices have a bundled type called \verb;my_bundled_vertex;
\end{quote}


